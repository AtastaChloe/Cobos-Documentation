\chapter{Compiling a local source on mainframe}

\mxproduct~is specially designed to easily develop mainframe applications. Also, you can compile a local program on the mainframe and retrieve the SYSOUT locally. 

The error messages are displayed in the Problems view synchronized with the source in the editor.

Z/Navigator plug-in is designed to process mainframe files directly from eclipse but requires scripts installation on mainframe.

If you are comfortable with JCL editing, try adapting the JCL sample supplied for a batch program. If not, feel free to go to next sequence: \nameref{chap:PDS} page \pageref{chap:PDS}.

In the PROJECT\_SAMPLE project:

\begin{enumerate}
\item Open \textit{Custload.cob} in COBOL Editor.

\item Open \textit{compcob.jcl} in JCL Editor
\mximg{1}{complocalsrc-01.png}\\[0.5ex]
\textit{Custload} program requires neither CICS nor SQL. So, you can leave the comment lines in STEPLIB.\\
Check the COBOL STEPLIB and, if necessary adapt the DSNAME according with the version of Enterprise COBOL in use (here IGY420 qualifier for COBOL 4.2 compiler).
\mximg{1}{complocalsrc-02.png}\\[0.5ex]
\textit{Custload} program does not require copies and object modules. So, leave the comment lines in SYSLIB (COBOL and LINKED).

\item Select Custload.cob tab (\mximgtext{complocalsrc-03.png}).

\item Double-click on \textbf{``1. FTP access \RHD~6. Compile local source''} in the Commands view:
\mximage{complocalsrc-04.png}

\item Fill in the form fields and select \textit{compcob.jcl} as JCL template with the ``\textbf{browse}'' button:
\mximage{complocalsrc-05.png}

\item Click on ``\textbf{OK}'' button to launch the compilation.

\item See the result in the ``\textbf{Problems}'' view:
\mxlargeimage{complocalsrc-06.png}
You should see a warning at line 45.

\item On the warning line, Right-click and select ``Open Sysout'' in context menu:
\mximage{complocalsrc-07.png}
The sysout opens in the text editor and the warning line is selected.
\end{enumerate}
