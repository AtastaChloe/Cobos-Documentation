\chapter{Introduction}

The purpose of this guide is to explain how to install, configure and use the Eclipse plug-ins \mxproduct.

You will edit and compile both local COBOL programs and mainframe COBOL programs WITHOUT INSTALLATION OF MAINFRAME RESOURCES.

\section{What's new}

\textbf{Improvements in \mxproduct~\mxversion{:}}

\begin{itemize}
\item \mxproduct~\mxversion~is fully qualified with the very latest Eclipse version such as Kepler (4.3) and Luna (4.4).

\item Support of listings coming from the remote Micro Focus\textregistered~compiler:\\
the COBOL programs and copybooks are marked with the compilation messages in the Problems view.

\item Preprocessing capabilities:
\begin{itemize}
\item Custom processing can be called before the GNU Cobol Check Syntax and Unfolding.

\item A standard post processing is proposed so that the messages are marked in the right place in the source code.
\end{itemize}

(a support of custom macro instructions has been implemented and distributed as a custom plug-in)
\end{itemize}

\vspace{0.5cm}
\textbf{This release solves the following bugs:}


\begin{description}
\item[5320] FTP Access supports use of non standard TCP/IP port

\end{description}
\newpage

\section{Prerequisites}

Ensure that the workstation has at least 2GB of RAM.

Supported OS: Windows XP SP3, Windows 7.

Ensure that a Java JRE 6 or 7 is present on the workstation.(JRE 8 is not yet fully qualified.)

This \mxproduct~\mxversion~Release must be installed on Eclipse Helios 3.6.2, Indigo 3.7.2, Kepler 4.3.x or Luna 4.4.x (32bits or 64bits).

\textbf{FTP Access module requires installation of a REXX interpreter on the workstation }such as
\href{http://sourceforge.net/projects/regina-rexx/files/regina-rexx/}{Regina REXX Interpreter} (version 3.6 or 3.8.2 recommended)\footnote{
For the users of Open Object Rexx (ooRexx), some scripts of \mxproduct~are not fully compatible.}.
\\[1.5ex]

Resources:

You downloaded \textbf{\mxproduct\_\mxversion.x\_Release-demo.zip }
\ifthenelse{\equal{\mxproduct}{Cobos}}{
or \textbf{\mxproduct\_\mxversion.x\_Essentials-demo.zip}} from the
\href{\mxurl}{\mxproduct~site}.

Once you have unzipped the downloaded file, you've got the following directories and files:

\begin{itemize}
\item \textbf{Demo\_workspace} which contains preconfigured resources for this ``quickstart''

\item \textbf{Products} which contains the plug-ins

\item \textbf{\mxproduct\_V\mxversion\_DemoReadme.txt}

\item \textbf{\mxproduct\_V\mxversion\_Quickstart.pdf}: this file!

\item \textbf{\mxproduct\_V\mxversion\_User\_Guide.pdf}

\item \textbf{releaseNotes.txt}
\end{itemize}
